%----------------------------------------
%-----     Enthält alle Pakete     -----
%----------------------------------------

\usepackage[ngerman]{babel} 
\usepackage[utf8]{inputenc}              
\usepackage[T1]{fontenc}
\usepackage{microtype}
\usepackage[numbers]{natbib}
%%%% Hier kann zwischen Schtiftarten umgestellt werden
\usepackage{palatino}
\usepackage{nth}
\usepackage{calc}  
\usepackage{enumitem} 
\setlist[description]{leftmargin=\parindent,labelindent=\parindent}
%\usepackage{lmodern}
\usepackage[format=plain,
			indention=0cm]{caption}




%===============================================================================
%%% Pakete für mathematische Formeln
\usepackage{amsmath}
\usepackage{amsfonts}
\usepackage{amssymb}
\usepackage{bbm}
\usepackage[output-decimal-marker={,}, detect-all=true]{siunitx} %anpassen bei englischer Version
\usepackage{calc}


\makeatletter
\newcommand*\@dblLabelI {}
\newcommand*\@dblLabelII {}
\newcommand*\@dblequationAux {}


%===============================================================================
%%% Pakete für Grafiken und Positionierung
\usepackage{graphicx}
\usepackage{tikz}
\usepackage{here}
\usepackage{pgfplots}
\usepackage[absolute]{textpos}

%===============================================================================
%%% Tikz-Libraries
\usetikzlibrary{
	external,
	calc,trees,shadows,positioning,arrows,chains,
	decorations.pathreplacing,
	decorations.pathmorphing,
	decorations.shapes,
	decorations.text,
	shapes,
	shapes.geometric,
	shapes.symbols,
	matrix,
	patterns,
	intersections,
	fit
}


\pgfplotsset{compat=newest}

%===============================================================================
%%% Pakete für Tabellen und Abbildungen
\usepackage{tabularx}
%\usepackage{ltablex}
\usepackage{colortbl}
\usepackage{color}
\usepackage{makecell}%To keep spacing of text in tables
\setcellgapes{4pt}%parameter for the spacing
\usepackage{fancyvrb}
\usepackage{verbdef}  % using verbatim in tables



%===============================================================================
%%% Neue Befehle 
\newcommand{\clearemptydoublepage}{\newpage{\pagestyle{empty}\cleardoublepage}}

% \monthword{#} übersetzt die Zahlen 1-12 in Monate
\newcommand{\monthword}[1]{\ifcase#1\or Januar\or Februar\or M\"arz\or April\or
                                        Mai\or Juni\or Juli\or August\or
                                        September\or Oktober\or November\or Dezember\fi} 

%% Legt den beim \smallskip-Befehl einzufügenden Abstand explizit fest
% \setlength{\smallskipamount}{Abstand}                  
% \setlength{\smallskipamount}{4mm}



%===============================================================================
%%% Formatierung Untertitel von Tabellen und Abbildungen
\usepackage[figurename={Abb.}, tablename={Tab.}]{caption}


%===============================================================================
%%% Seitenlayout
\usepackage{geometry}
\textheight230mm		% Seitenformatierung siehe fancyhdr-Dokumentation Seite 3
\textwidth160mm			% http://tug.ctan.org/macros/latex/contrib/fancyhdr/fancyhdr.pdf
\voffset-5mm
\hoffset-25mm
\evensidemargin25mm
\oddsidemargin25mm	
\parindent0mm			% Einrücktiefe der ersten Zeile eines Absatzes
\parskip1mm				% Legt den Abstand zwischen den nachfolgenden Absätzen fest. 
\fboxrule0.1mm			% Definiert die Linienstärke für nachfolgende fbox- und framebox-Befehle
\fboxsep1mm				% Definiert den Abstand zwischen dem Rahmen einer fbox (oder framebox) und ihrem Inhalt


%===============================================================================
%%% Kopf- und Fußzeile
\usepackage{fancyhdr}
\pagestyle{fancy}
\fancyhf{}
\renewcommand{\headrulewidth}{0.pt}
\renewcommand{\footrulewidth}{0.pt}
\setlength\headheight{16pt}
\makeatletter
\renewcommand{\chaptermark}[1]{\markboth{\thechapter\ #1}{}}	% Format des chapter in Kopfzeile
\renewcommand{\sectionmark}[1]{\markright{\thesection\ #1}}		% Format der section in Kopfzeile

	\fancypagestyle{empty}{
		\fancyhf{}
		\renewcommand{\headrulewidth}{0.pt}
		\renewcommand{\footrulewidth}{0.pt}
	}

	%================================================================================
	%%% Kopf- und Fußzeile auf Kapitelseiten
	\fancypagestyle{plain}{%
		\renewcommand{\headrulewidth}{0pt}%
		\renewcommand{\footrulewidth}{0.1pt}%
		\fancyhf{}%
		\fancyfoot[OR]{Seite \thepage}				% Seitenzahl
		\fancyfoot[EL]{Seite \thepage} 				% Seitenzahl
	}

	%===============================================================================
	%%% Kopf- und Fußzeile auf ersten Seiten
	\fancypagestyle{titelblatt}{
	\fancyhead{}
	\fancyfoot{}
		\fancyhead[OL]{\setlength{\unitlength}{1mm}
						\begin{picture}(0,0)
							\put(0,-30){\includegraphics[width=40mm]{Bilder/0_Deckblatt/KIT_logo}}
						\end{picture}
					  }

		\fancyhead[OR]{\begin{picture}(0,0)
							\put(-152,-60){\parbox{15em}{\linespread{1}\scriptsize\textbf{Institut für Fahrzeugsystemtechnik}\\
							\textbf{Leichtbautechnologie}\\
							\scriptsize Leiter: Prof. Dr.-Ing. Frank Henning \smallskip \smallskip
							\ \\ 
							Rintheimer Querallee 2\\
							76131 Karlsruhe
							}}
							\end{picture}
					  }
					  
		\fancyfoot[OL]{
					  {\vspace{1,0cm}\hspace{-0,7cm}\tiny KIT -- Die Forschungsuniversität in der Helmholtz-Gemeinschaft}}
					  
		\fancyfoot[OR]{
					  \vspace{1,0cm} www.kit.edu\hspace{-0,7cm}}
					  
		\begin{tikzpicture}[remember picture, overlay]
			\centering
			\begin{tikzpicture}[remember picture, overlay,shift={(80mm,-115mm)}]
				%\draw(95mm,-128.5mm) -- (95mm,133.5mm) arc (0:90:5mm) -- (-95mm,138.5mm) -- (-95mm,-123.5mm) arc (180:270:5mm) -- cycle; 
				%\draw[red](80mm,115mm)--(80mm,-115mm)--(-80mm,-115mm)--(-80mm,115mm)--(80mm,115mm);
				\draw(92.5mm,-127mm) -- (92.5mm,130mm)arc (0:90:5mm) --(-92.5mm,135mm) --(-92.5mm,-122mm) arc(180:270:5mm) -- (92.5mm,-127mm);
			\end{tikzpicture};
		\end{tikzpicture}	
	}
%===============================================================================
	%%% Kopf- und Fußzeile in der Präambel
	\fancypagestyle{preamble}{
		\fancyhf{}
		\fancyhead{}
		\fancyfoot{}
		\renewcommand{\headrulewidth}{0pt}		% Dicke des Strichs oben
		\renewcommand{\footrulewidth}{0.1pt}			% Dicke des Strichs unten
		
		\fancyfoot[OR]{Seite \thepage}				% Seitenzahl
		\fancyfoot[EL]{Seite \thepage} 				% Seitenzahl
	}

	%===============================================================================
	%%% Kopf- und Fußzeile im Hauptteil
	\fancypagestyle{hauptteil}{
		\fancyhead{}
		\fancyfoot{}
		\renewcommand{\headrulewidth}{0.1pt}		% Dicke des Strichs oben
		\renewcommand{\footrulewidth}{0.1pt}			% Dicke des Strichs unten
	
		% E Even page; O Odd page; L Left field; C Center field; R Right field;
		% \fancyhead[...,...]{} oder \fancyfoot[...,...]{}
		\fancyhead[EL]{\nouppercase \leftmark}		% Chapter
		\fancyhead[EC]{}
		\fancyhead[ER]{}
	
		\fancyhead[OL]{}
		\fancyhead[OC]{}
		\fancyhead[OR]{\nouppercase \rightmark}		% Section
		
		\fancyfoot[OR]{Seite \thepage}				% Seitenzahl
		\fancyfoot[EL]{Seite \thepage} 				% Seitenzahl	
	}

%===============================================================================
%%% Zeilenabstand
\renewcommand{\baselinestretch}{1.2} % Faktor, der mit dem Zeilenabstand (baselineskip) aufmultipliziert wird und so den effektiven Zeilenabstand

%===============================================================================
%%% Worttrennung ausschalten
%\sloppy						% Schaltet auf eine großzügige Formatierungsweise um, die relativ wenige Worttrennungen am Zeilenende erzeugt
%\hyphenpenalty=10000			% To avoid hyphenation altogether (extreme penalty value)
%\global\hyphenpenalty=100000	% global assignment
%\exhyphenpenalty=10000			% Penalty for hyphenating a word which already contains a hyphen


%===============================================================================
%%% Kapitelüberschrift (Jetzt als: Kapitel 4 <0.8cm space> [...] )
\makeatletter
\def\@makechapterhead#1{%
  \vspace*{50\p@}%
  {\parindent \z@ \raggedright \normalfont
    \ifnum \c@secnumdepth >\m@ne
      \if@mainmatter
        %\Huge\bfseries \@chapapp\space \thechapter
        \Huge\bfseries \thechapter \hspace{0.8cm}%
        %\par\nobreak
        %\vskip 20\p@
      \fi
    \fi
    \interlinepenalty\@M
    \Huge \bfseries #1\par\nobreak
    \vskip 40\p@
  }}
\makeatother


%===============================================================================
%%% Sonstige Pakete und Funktionen
%\usepackage{nomencl}				% Symbolverzeichnis
\usepackage{appendix} 				% Anhang und Formatierung im Inhaltsverzeichnis
\usepackage{geometry}
\usepackage{ifthen}
\usepackage{setspace}
\usepackage[all]{nowidow} %verhindert Hurenkinder

%===============================================================================
%%% Farben
\definecolor{hellgrau}{rgb}{0.949,0.949,0.949}
\definecolor{lightgray}{RGB}{231,230,230} 
\definecolor{cadmiumgreen}{rgb}{0.0,0.42,0.24}
\definecolor{CorpLightBlue}{RGB}{82,141,193} 
\definecolor{CorpOrange}{RGB}{237,125,49} 
\definecolor{CorpRed}{RGB}{194,19,15}
\definecolor{CorpBlue}{RGB}{20,112,192}  
\definecolor{tuerkis}{RGB}{36,163,178} 
\definecolor{DarkOrange}{RGB}{221,117,42} 
 


%===============================================================================
%%% PDF-Optionen
\usepackage[breaklinks=true,citebordercolor={1 1 1},linkbordercolor={1 1 1}, hidelinks]{hyperref}
\hypersetup{bookmarksopen,	        		% Anzeige der Bookmarks mit allen Untereintraegen
            bookmarksopenlevel=1,			% Verschachtelungstiefe der angezeigten Bookmarks
            bookmarksnumbered,	   			% Bookmarks enthalten Ueberschriftsnummern
            pdfhighlight=/I,	      		% Aussehen des Link-Buttons beim Druecken
            pdfpagemode=UseOutlines,		% Legt fest, wie das Dokument geoeffnet werden soll
            german, 	              		% Deutsche Bezeichnung der Hyperlinks            
            pdflang=ger,            		% PDF-Sprachkennung nach RFC 3066, ger - deutsch
         %% pdfstartview=FitV, 	  			% Anzeigemodus fuer die Startseite
            pdfdisplaydoctitle,  	  		% Anzeige des Dokumenttitels in der Titelleiste
}


%%% selbst eingebundene Pakete%%%%

\usepackage{mwe}
\usepackage{graphbox}
\usepackage{mathtools}
\usepackage{tikz-3dplot}
\usepackage{tkz-euclide}
\def\centerarc[#1](#2)(#3:#4:#5){	\draw[#1] ($(#2) + ({#5*cos(#3)},{#5*sin(#3)})$) arc (#3:#4:#5);}
\usepackage{multirow}
\usepackage[normalem]{ulem}
\usepackage{booktabs}

\usepackage{listings}
\usepackage{setspace}
\definecolor{Code}{rgb}{0,0,0}
\definecolor{Decorators}{rgb}{0.5,0.5,0.5}
\definecolor{Numbers}{rgb}{0.5,0,0}
\definecolor{MatchingBrackets}{rgb}{0.25,0.5,0.5}
\definecolor{Keywords}{rgb}{0,0,1}
\definecolor{self}{rgb}{0,0,0}
\definecolor{Strings}{rgb}{0,0.63,0}
%\definecolor{Comments}{rgb}{0,0.63,1}
\definecolor{Comments}{RGB}{28,131,31}
\definecolor{Backquotes}{rgb}{0,0,0}
\definecolor{Classname}{rgb}{0,0,0}
\definecolor{FunctionName}{rgb}{0,0,0}
\definecolor{Operators}{rgb}{0,0,0}
\definecolor{Background}{rgb}{0.98,0.98,0.98}
\lstdefinestyle{Python222}{
	language=Python,
	breaklines=true,
	postbreak=\mbox{\textcolor{red}{$\hookrightarrow$}\space},
	numbers=left,
	numberstyle=\footnotesize,
	numbersep=1em,
	xleftmargin=1em,
	framextopmargin=2em,
	framexbottommargin=2em,
	showspaces=false,
	showtabs=false,
	showstringspaces=false,
	frame=l,
	tabsize=4,
	% Basic
	basicstyle=\ttfamily\small\setstretch{1},
	backgroundcolor=\color{Background},
	% Comments
%	commentstyle=\color{Comments}\slshape,
	commentstyle=\color{Comments},
	% Strings
	stringstyle=\color{darkgray},
	morecomment=[s][\color{Strings}]{"""}{"""},
	morecomment=[s][\color{Strings}]{'''}{'''},
	% keywords
	morekeywords={import,from,class,def,for,while,if,is,in,elif,else,not,and,or,print,break,continue,return,True,False,None,access,as,,del,except,exec,finally,global,import,lambda,pass,print,raise,try,assert},
	keywordstyle={\color{Keywords}\bfseries},
	% additional keywords
	%morekeywords={[2]@invariant,pylab,numpy,np,scipy},
	keywordstyle={[2]\color{Decorators}\slshape},
	emph={self},
	emphstyle={\color{self}\slshape},
	%
}
\linespread{1.3}


\usepackage{threeparttable,booktabs}
\usepackage[section]{placeins}
\usepackage{longtable}

%% sorgt dafuer, dass Indizes auch im math-mode automatisch nichtkursiv gesetzt werden. 
\def\subinrm#1{\sb{\rm#1}}
{\catcode`\_=13 \global\let_=\subinrm}
\mathcode`_="8000
\def\upsubscripts{\catcode`\_=12 } \def\normalsubscripts{\catcode`\_=8 }

\upsubscripts




