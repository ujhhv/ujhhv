\documentclass[12pt,a4paper,twoside]{book}
\newcounter{NstopRoman}

%===============================================================================
%%% Einlesen des Formats und aller notwendiger Pakete
%----------------------------------------
%-----     Enthält alle Pakete     -----
%----------------------------------------

\usepackage[ngerman]{babel} 
\usepackage[utf8]{inputenc}              
\usepackage[T1]{fontenc}
\usepackage{microtype}
\usepackage[numbers]{natbib}
%%%% Hier kann zwischen Schtiftarten umgestellt werden
\usepackage{palatino}
\usepackage{nth}
\usepackage{calc}  
\usepackage{enumitem} 
\setlist[description]{leftmargin=\parindent,labelindent=\parindent}
%\usepackage{lmodern}
\usepackage[format=plain,
			indention=0cm]{caption}




%===============================================================================
%%% Pakete für mathematische Formeln
\usepackage{amsmath}
\usepackage{amsfonts}
\usepackage{amssymb}
\usepackage{bbm}
\usepackage[output-decimal-marker={,}, detect-all=true]{siunitx} %anpassen bei englischer Version
\usepackage{calc}


\makeatletter
\newcommand*\@dblLabelI {}
\newcommand*\@dblLabelII {}
\newcommand*\@dblequationAux {}


%===============================================================================
%%% Pakete für Grafiken und Positionierung
\usepackage{graphicx}
\usepackage{tikz}
\usepackage{here}
\usepackage{pgfplots}
\usepackage[absolute]{textpos}

%===============================================================================
%%% Tikz-Libraries
\usetikzlibrary{
	external,
	calc,trees,shadows,positioning,arrows,chains,
	decorations.pathreplacing,
	decorations.pathmorphing,
	decorations.shapes,
	decorations.text,
	shapes,
	shapes.geometric,
	shapes.symbols,
	matrix,
	patterns,
	intersections,
	fit
}


\pgfplotsset{compat=newest}

%===============================================================================
%%% Pakete für Tabellen und Abbildungen
\usepackage{tabularx}
%\usepackage{ltablex}
\usepackage{colortbl}
\usepackage{color}
\usepackage{makecell}%To keep spacing of text in tables
\setcellgapes{4pt}%parameter for the spacing
\usepackage{fancyvrb}
\usepackage{verbdef}  % using verbatim in tables



%===============================================================================
%%% Neue Befehle 
\newcommand{\clearemptydoublepage}{\newpage{\pagestyle{empty}\cleardoublepage}}

% \monthword{#} übersetzt die Zahlen 1-12 in Monate
\newcommand{\monthword}[1]{\ifcase#1\or Januar\or Februar\or M\"arz\or April\or
                                        Mai\or Juni\or Juli\or August\or
                                        September\or Oktober\or November\or Dezember\fi} 

%% Legt den beim \smallskip-Befehl einzufügenden Abstand explizit fest
% \setlength{\smallskipamount}{Abstand}                  
% \setlength{\smallskipamount}{4mm}



%===============================================================================
%%% Formatierung Untertitel von Tabellen und Abbildungen
\usepackage[figurename={Abb.}, tablename={Tab.}]{caption}


%===============================================================================
%%% Seitenlayout
\usepackage{geometry}
\textheight230mm		% Seitenformatierung siehe fancyhdr-Dokumentation Seite 3
\textwidth160mm			% http://tug.ctan.org/macros/latex/contrib/fancyhdr/fancyhdr.pdf
\voffset-5mm
\hoffset-25mm
\evensidemargin25mm
\oddsidemargin25mm	
\parindent0mm			% Einrücktiefe der ersten Zeile eines Absatzes
\parskip1mm				% Legt den Abstand zwischen den nachfolgenden Absätzen fest. 
\fboxrule0.1mm			% Definiert die Linienstärke für nachfolgende fbox- und framebox-Befehle
\fboxsep1mm				% Definiert den Abstand zwischen dem Rahmen einer fbox (oder framebox) und ihrem Inhalt


%===============================================================================
%%% Kopf- und Fußzeile
\usepackage{fancyhdr}
\pagestyle{fancy}
\fancyhf{}
\renewcommand{\headrulewidth}{0.pt}
\renewcommand{\footrulewidth}{0.pt}
\setlength\headheight{16pt}
\makeatletter
\renewcommand{\chaptermark}[1]{\markboth{\thechapter\ #1}{}}	% Format des chapter in Kopfzeile
\renewcommand{\sectionmark}[1]{\markright{\thesection\ #1}}		% Format der section in Kopfzeile

	\fancypagestyle{empty}{
		\fancyhf{}
		\renewcommand{\headrulewidth}{0.pt}
		\renewcommand{\footrulewidth}{0.pt}
	}

	%================================================================================
	%%% Kopf- und Fußzeile auf Kapitelseiten
	\fancypagestyle{plain}{%
		\renewcommand{\headrulewidth}{0pt}%
		\renewcommand{\footrulewidth}{0.1pt}%
		\fancyhf{}%
		\fancyfoot[OR]{Seite \thepage}				% Seitenzahl
		\fancyfoot[EL]{Seite \thepage} 				% Seitenzahl
	}

	%===============================================================================
	%%% Kopf- und Fußzeile auf ersten Seiten
	\fancypagestyle{titelblatt}{
	\fancyhead{}
	\fancyfoot{}
		\fancyhead[OL]{\setlength{\unitlength}{1mm}
						\begin{picture}(0,0)
							\put(0,-30){\includegraphics[width=40mm]{Bilder/0_Deckblatt/KIT_logo}}
						\end{picture}
					  }

		\fancyhead[OR]{\begin{picture}(0,0)
							\put(-152,-60){\parbox{15em}{\linespread{1}\scriptsize\textbf{Institut für Fahrzeugsystemtechnik}\\
							\textbf{Leichtbautechnologie}\\
							\scriptsize Leiter: Prof. Dr.-Ing. Frank Henning \smallskip \smallskip
							\ \\ 
							Rintheimer Querallee 2\\
							76131 Karlsruhe
							}}
							\end{picture}
					  }
					  
		\fancyfoot[OL]{
					  {\vspace{1,0cm}\hspace{-0,7cm}\tiny KIT -- Die Forschungsuniversität in der Helmholtz-Gemeinschaft}}
					  
		\fancyfoot[OR]{
					  \vspace{1,0cm} www.kit.edu\hspace{-0,7cm}}
					  
		\begin{tikzpicture}[remember picture, overlay]
			\centering
			\begin{tikzpicture}[remember picture, overlay,shift={(80mm,-115mm)}]
				%\draw(95mm,-128.5mm) -- (95mm,133.5mm) arc (0:90:5mm) -- (-95mm,138.5mm) -- (-95mm,-123.5mm) arc (180:270:5mm) -- cycle; 
				%\draw[red](80mm,115mm)--(80mm,-115mm)--(-80mm,-115mm)--(-80mm,115mm)--(80mm,115mm);
				\draw(92.5mm,-127mm) -- (92.5mm,130mm)arc (0:90:5mm) --(-92.5mm,135mm) --(-92.5mm,-122mm) arc(180:270:5mm) -- (92.5mm,-127mm);
			\end{tikzpicture};
		\end{tikzpicture}	
	}
%===============================================================================
	%%% Kopf- und Fußzeile in der Präambel
	\fancypagestyle{preamble}{
		\fancyhf{}
		\fancyhead{}
		\fancyfoot{}
		\renewcommand{\headrulewidth}{0pt}		% Dicke des Strichs oben
		\renewcommand{\footrulewidth}{0.1pt}			% Dicke des Strichs unten
		
		\fancyfoot[OR]{Seite \thepage}				% Seitenzahl
		\fancyfoot[EL]{Seite \thepage} 				% Seitenzahl
	}

	%===============================================================================
	%%% Kopf- und Fußzeile im Hauptteil
	\fancypagestyle{hauptteil}{
		\fancyhead{}
		\fancyfoot{}
		\renewcommand{\headrulewidth}{0.1pt}		% Dicke des Strichs oben
		\renewcommand{\footrulewidth}{0.1pt}			% Dicke des Strichs unten
	
		% E Even page; O Odd page; L Left field; C Center field; R Right field;
		% \fancyhead[...,...]{} oder \fancyfoot[...,...]{}
		\fancyhead[EL]{\nouppercase \leftmark}		% Chapter
		\fancyhead[EC]{}
		\fancyhead[ER]{}
	
		\fancyhead[OL]{}
		\fancyhead[OC]{}
		\fancyhead[OR]{\nouppercase \rightmark}		% Section
		
		\fancyfoot[OR]{Seite \thepage}				% Seitenzahl
		\fancyfoot[EL]{Seite \thepage} 				% Seitenzahl	
	}

%===============================================================================
%%% Zeilenabstand
\renewcommand{\baselinestretch}{1.2} % Faktor, der mit dem Zeilenabstand (baselineskip) aufmultipliziert wird und so den effektiven Zeilenabstand

%===============================================================================
%%% Worttrennung ausschalten
%\sloppy						% Schaltet auf eine großzügige Formatierungsweise um, die relativ wenige Worttrennungen am Zeilenende erzeugt
%\hyphenpenalty=10000			% To avoid hyphenation altogether (extreme penalty value)
%\global\hyphenpenalty=100000	% global assignment
%\exhyphenpenalty=10000			% Penalty for hyphenating a word which already contains a hyphen


%===============================================================================
%%% Kapitelüberschrift (Jetzt als: Kapitel 4 <0.8cm space> [...] )
\makeatletter
\def\@makechapterhead#1{%
  \vspace*{50\p@}%
  {\parindent \z@ \raggedright \normalfont
    \ifnum \c@secnumdepth >\m@ne
      \if@mainmatter
        %\Huge\bfseries \@chapapp\space \thechapter
        \Huge\bfseries \thechapter \hspace{0.8cm}%
        %\par\nobreak
        %\vskip 20\p@
      \fi
    \fi
    \interlinepenalty\@M
    \Huge \bfseries #1\par\nobreak
    \vskip 40\p@
  }}
\makeatother


%===============================================================================
%%% Sonstige Pakete und Funktionen
%\usepackage{nomencl}				% Symbolverzeichnis
\usepackage{appendix} 				% Anhang und Formatierung im Inhaltsverzeichnis
\usepackage{geometry}
\usepackage{ifthen}
\usepackage{setspace}
\usepackage[all]{nowidow} %verhindert Hurenkinder

%===============================================================================
%%% Farben
\definecolor{hellgrau}{rgb}{0.949,0.949,0.949}
\definecolor{lightgray}{RGB}{231,230,230} 
\definecolor{cadmiumgreen}{rgb}{0.0,0.42,0.24}
\definecolor{CorpLightBlue}{RGB}{82,141,193} 
\definecolor{CorpOrange}{RGB}{237,125,49} 
\definecolor{CorpRed}{RGB}{194,19,15}
\definecolor{CorpBlue}{RGB}{20,112,192}  
\definecolor{tuerkis}{RGB}{36,163,178} 
\definecolor{DarkOrange}{RGB}{221,117,42} 
 


%===============================================================================
%%% PDF-Optionen
\usepackage[breaklinks=true,citebordercolor={1 1 1},linkbordercolor={1 1 1}, hidelinks]{hyperref}
\hypersetup{bookmarksopen,	        		% Anzeige der Bookmarks mit allen Untereintraegen
            bookmarksopenlevel=1,			% Verschachtelungstiefe der angezeigten Bookmarks
            bookmarksnumbered,	   			% Bookmarks enthalten Ueberschriftsnummern
            pdfhighlight=/I,	      		% Aussehen des Link-Buttons beim Druecken
            pdfpagemode=UseOutlines,		% Legt fest, wie das Dokument geoeffnet werden soll
            german, 	              		% Deutsche Bezeichnung der Hyperlinks            
            pdflang=ger,            		% PDF-Sprachkennung nach RFC 3066, ger - deutsch
         %% pdfstartview=FitV, 	  			% Anzeigemodus fuer die Startseite
            pdfdisplaydoctitle,  	  		% Anzeige des Dokumenttitels in der Titelleiste
}


%%% selbst eingebundene Pakete%%%%

\usepackage{mwe}
\usepackage{graphbox}
\usepackage{mathtools}
\usepackage{tikz-3dplot}
\usepackage{tkz-euclide}
\def\centerarc[#1](#2)(#3:#4:#5){	\draw[#1] ($(#2) + ({#5*cos(#3)},{#5*sin(#3)})$) arc (#3:#4:#5);}
\usepackage{multirow}
\usepackage[normalem]{ulem}
\usepackage{booktabs}

\usepackage{listings}
\usepackage{setspace}
\definecolor{Code}{rgb}{0,0,0}
\definecolor{Decorators}{rgb}{0.5,0.5,0.5}
\definecolor{Numbers}{rgb}{0.5,0,0}
\definecolor{MatchingBrackets}{rgb}{0.25,0.5,0.5}
\definecolor{Keywords}{rgb}{0,0,1}
\definecolor{self}{rgb}{0,0,0}
\definecolor{Strings}{rgb}{0,0.63,0}
%\definecolor{Comments}{rgb}{0,0.63,1}
\definecolor{Comments}{RGB}{28,131,31}
\definecolor{Backquotes}{rgb}{0,0,0}
\definecolor{Classname}{rgb}{0,0,0}
\definecolor{FunctionName}{rgb}{0,0,0}
\definecolor{Operators}{rgb}{0,0,0}
\definecolor{Background}{rgb}{0.98,0.98,0.98}
\lstdefinestyle{Python222}{
	language=Python,
	breaklines=true,
	postbreak=\mbox{\textcolor{red}{$\hookrightarrow$}\space},
	numbers=left,
	numberstyle=\footnotesize,
	numbersep=1em,
	xleftmargin=1em,
	framextopmargin=2em,
	framexbottommargin=2em,
	showspaces=false,
	showtabs=false,
	showstringspaces=false,
	frame=l,
	tabsize=4,
	% Basic
	basicstyle=\ttfamily\small\setstretch{1},
	backgroundcolor=\color{Background},
	% Comments
%	commentstyle=\color{Comments}\slshape,
	commentstyle=\color{Comments},
	% Strings
	stringstyle=\color{darkgray},
	morecomment=[s][\color{Strings}]{"""}{"""},
	morecomment=[s][\color{Strings}]{'''}{'''},
	% keywords
	morekeywords={import,from,class,def,for,while,if,is,in,elif,else,not,and,or,print,break,continue,return,True,False,None,access,as,,del,except,exec,finally,global,import,lambda,pass,print,raise,try,assert},
	keywordstyle={\color{Keywords}\bfseries},
	% additional keywords
	%morekeywords={[2]@invariant,pylab,numpy,np,scipy},
	keywordstyle={[2]\color{Decorators}\slshape},
	emph={self},
	emphstyle={\color{self}\slshape},
	%
}
\linespread{1.3}


\usepackage{threeparttable,booktabs}
\usepackage[section]{placeins}
\usepackage{longtable}

%% sorgt dafuer, dass Indizes auch im math-mode automatisch nichtkursiv gesetzt werden. 
\def\subinrm#1{\sb{\rm#1}}
{\catcode`\_=13 \global\let_=\subinrm}
\mathcode`_="8000
\def\upsubscripts{\catcode`\_=12 } \def\normalsubscripts{\catcode`\_=8 }

\upsubscripts






%% set options for tikzexternalize
\tikzexternalize[shell escape=-shell-escape]
\tikzexternalize[shell escape=pdflatex -shell-escape] %I used this method as well
\tikzexternalize
\tikzsetexternalprefix{tikz/}

%===============================================================================
%%% Einlesen der Symbole und Hotkeys
\input{Makros_BScMSc}


%===============================================================================
%%% Informationen zum Autor, Titel, usw.
\newcommand{\Author}{Der Autor}
\newcommand{\Title}{Der Titel}
\newcommand{\EnglishTitle}{Der englische Titel}
\newcommand{\matrnr}{Die Matrikelnummer}
\newboolean{authorismale}
\setboolean{authorismale}{true}

\author{\Author}
\title{\Title}


\begin{document}

	\frontmatter
	\tikzexternaldisable

%===============================================================================
%%% Titelblatt der Arbeit
	%\begin{titlepage}
	\pagestyle{titelblatt}
	\vspace*{4cm}
	\begin{center}
		{\huge Masterarbeit}
		\vspace{2cm}

		{\Large \Title}
		\vspace{2cm}

		{\large \Author}
		\vspace{2cm}

		%\includegraphics[width=.8\textwidth]{Bilder/0_Deckblatt/frontpage.pdf}
		Hier kommt dann das Titelbild hin

		%% v-space selbst anpassen
		\vspace{6.5cm}

	\end{center}
		\begin{tabularx}{\textwidth}{lX}
		Projektleiter: & Der Projektleiter \\ & Leichtbautechnologie \\
		& \\
		Nr.: JJ-L-XXXX & \multicolumn{1}{r}{Karlsruhe, Der Monat \the\year} \\
		\end{tabularx}
	%\end{titlepage}

%===============================================================================
%%% Unterschriftsblatt
   	\clearemptydoublepage
	\pagestyle{titelblatt}
	\vspace*{2cm}
	\begin{center}
		{\Large \Title\\}
		\vspace{.2cm}
		{\Large \EnglishTitle}

		{\large \Author\\
		Matr.-Nr.: \matrnr}
		\vspace{.2cm}

		\vfill
		
		\small Die Abschlussarbeit ist im engen Kontakt mit dem Institut auszuarbeiten.
		Alle dem Institut übergebenen Exemplare werden Eigentum des Instituts.
		Eine Einsichtnahme Dritter in diese Exemplare darf nur nach Rücksprache mit dem Institut erfolgen.
		
		\vspace{.2cm}

		\begin{tabular}[b]{p{2,25cm}p{3,75cm}p{1,8cm}p{2,25cm}p{3,75cm}}
		Ausgabetag: & xx.yy.zzzz & & Abgabetag: & xx.yy.zzzz \\
		& & & & \\
		Betreuer: & & & Projektleiter: & \\
		& & & & \\[16mm]
		\cline{1-2} \cline{4-5}
		\multicolumn{2}{c}{(Prof. Dr.-Ing. Frank Henning)} & & \multicolumn{2}{c}{(Projektleiter)} \\
		& & & & \\[16mm]
		\cline{1-2} \cline{4-5}
		\multicolumn{2}{c}{(Dr.-Ing. Luise Kärger)} & & \multicolumn{2}{c}{(ggf. zweiter Projektleiter)} \\
		& & & & \\
		Bearbeiter:  & & & Anschrift:& \\
		& & &\ \ \ Ort & Der Ort \\
		& & &\ \ \ Straße & Die Straße\\
		& & &\ \ \ Telefon & Die Nummer\\[3mm]
		\cline{1-2}
		\multicolumn{2}{c}{(\Author)} & & & \\
		\end{tabular}
	\end{center}


%===============================================================================
%%% Aufgabenstellung ------------------------------------
	\clearemptydoublepage
	\pagestyle{titelblatt}
	\vspace*{2cm}
	\begin{center}

		{\large Aufgabenstellung\\
		für\\
		\ifthenelse{\boolean{authorismale}}{Herrn}{Frau}
		\Author\\
		(Matr.-Nr.: \matrnr)\par}
		\vspace{.2cm}

		{\Large \Title\\}
		\vspace{.2cm}
		{\Large \EnglishTitle}
		\vspace{1cm}

	\end{center}

	\noindent
	Text der Aufgabenstellung...


	\par




%===============================================================================
%%% Eigenständigkeitserklärung
	\clearemptydoublepage
	\pagestyle{preamble}
	%\addcontentsline{toc}{chapter}{Erklärung}     	%% Kapitel zum Inhaltsverzeichnis hinzugefügen
	\section*{Erklärung}
	Hiermit versichere ich, die vorliegende Arbeit selbständig und nur mit den im Literaturverzeichnis angegebenen Quellen und Hilfsmitteln angefertigt zu haben.\par

	\vspace{3cm}
	\makebox[10cm]{\hrulefill}\\
	\noindent Karlsruhe, den Der Tag, den Der Monat Das Jahr\par
	%\begin{tabular}{l}
	%	\hline
	%	\hspace{-3,5mm} Karlsruhe, den \the\day.~\monthword{\the\month}~\the\year\par
	%\end{tabular}

%===============================================================================

%%% Danksagung
	\clearemptydoublepage
	%\addcontentsline{toc}{chapter}{Danksagung}     	%% Kapitel zum Inhaltsverzeichnis hinzugefügen
	\section*{Danksagung}


	\par


%===============================================================================
%%% Kurzfassung
	\selectlanguage{ngerman}
	\clearemptydoublepage
	%\addcontentsline{toc}{chapter}{Kurzfassung}     	%% Kapitel zum Inhaltsverzeichnis hinzugefügen
	\section*{Kurzfassung}
	Eine Kurzfassung einer wissenschaftlichen Arbeit ist eine Kondensierung des Inhalts auf einen Absatz, höchstens eine Seite.
	Die Kurzfassung soll für sich stehen und die Motivation, Methoden, Ziele sowie Ergebnisse der Arbeit für jemanden präsentieren, der gerade keine Zeit hat, die ganze Arbeit zu lesen.
	Sie muss ihm eine Entscheidung darüber ermöglichen, ob er dieses Werk (nötigenfalls bestellen und) lesen muss. Der Kurzfassung kommt eine große Bedeutung zu und sollte sehr bewusst formuliert werden, da hier die entscheidenden Kernpunkte der Arbeit klar werden müssen.
	\begin{description}
		\item[Kontext:]{Der Leser soll zunächst abgeholt werden. Dazu wird in ein oder zwei Sätzen eingeordnet, mit welchem Themenfeld sich die Arbeit beschäftigt.}
		\item[Stand der Technik:]{An dieser Stelle wird sehr kurz der aktuelle Stand der Forschung zusammengefasst. Insbesondere sollten hier Unzulänglichkeiten bisheriger Ergebnisse adressiert werden, um auf die eigene Forschungsfrage hinzuarbeiten. }
		\item[Forschungslücke:]{Hier wird die eigene Forschungsfrage herausgearbeitet. Was fehlt bisher oder welche Aspekte müssen genauer untersucht werden?}
		\item[Lösungsansatz:]{Dieser Abschnitt beschreibt die eingesetzten Methoden - was ist der eigene Lösungsansatz für die Forschungsfrage?}
		\item[Ergebnisse:]{Was ist bei der Arbeit herausgekommen und welche Schlüsse können daraus gezogen werden?}
	\end{description}
	
	\par


%===============================================================================
%%% Abstract
	%\selectlanguage{english}
	\clearemptydoublepage
	%\addcontentsline{toc}{chapter}{Abstract}  %% Kapitel zum Inhaltsverzeichnis hinzugefügen
	\section*{Abstract}
	Englische Kurzfassung. Falls die Arbeit auf Englisch geschrieben wird, muss an dieser Stelle eine deutsche Kurzfassung eingefügt werden.\par

	\selectlanguage{ngerman}
	
%===============================================================================
%%% Inhaltsverzeichnis
	\clearemptydoublepage
	\pdfbookmark{\contentsname}{toc}
	\tableofcontents
	\clearemptydoublepage

%===============================================================================
%%% Abkürzungsverzeichnis
	\clearemptydoublepage
	\include{Kapitel/Abkuerzungen}

%===============================================================================
%%% Notation
	\clearemptydoublepage
	\include{Kapitel/Notation}
	\setcounter{NstopRoman}{\value{page}}
	

%===============================================================================
%%% Hauptteil
	\clearemptydoublepage
	\mainmatter				% Startet die arabische Nummerierung für den eigentlichen Inhalt des Dokuments
	%\pagenumbering{arabic}
	\tikzexternalenable
	%===============================================================================
	%%% Kopf- und Fußzeilen des Hauptteils
	\pagestyle{hauptteil}

	%===============================================================================
	%%% Kapitel
	\clearemptydoublepage
	\include{Kapitel/Demo}
	\clearemptydoublepage
	\include{Kapitel/Einleitung}
	\clearemptydoublepage
	\include{Kapitel/Kenntnisstand}
	\clearemptydoublepage
	\include{Kapitel/Methoden}
	\clearemptydoublepage
	\include{Kapitel/Ergebnisse}
	\clearemptydoublepage
	\include{Kapitel/Zusammenfassung_ausblick}


%===============================================================================
%%% Anhang
	\clearemptydoublepage
	\include{Kapitel/Anhang}

%===============================================================================
%%% Literaturverzeichnis
	\newpage
	\clearemptydoublepage
	\markboth{Literaturverzeichnis}{Literaturverzeichnis}
	\addcontentsline{toc}{chapter}{Literaturverzeichnis}   
	\bibliographystyle{unsrtnat_german} % Bei englischen Arbeiten einfach durch {unsrtnat}ersetzen
	\bibliography{lit}
	%------------------------------------------------------
\backmatter
\end{document}
